\documentclass[10pt,a4paper]{article}


\usepackage{tabularx}
\usepackage{float}
\usepackage{indent first}
\usepackage{hyperref}


\usepackage{draftwatermark}
\SetWatermarkText{}
\SetWatermarkScale{4}

\usepackage{geometry}
	\geometry{
		a4paper,
		left=0.75in,
		top=0.75in,
	}

\usepackage{array}
\newcolumntype{P}[1]{>{\raggedleft\arraybackslash}p{#1}}
\linespread{1.5}




\begin{document}

%----------------------------------------------------------------------------------------
%	HEADING SECTION
%----------------------------------------------------------------------------------------

\begin{figure}[H]
\begin{center}
	{\Large\bf Elena Gramellini} \\
\fontsize{10pt}{10pt}\selectfont  uuuu \\
	Phone:  Email: \href{mailto:elena.gramellini@yale.edu}{elena.gramellini@yale.edu}\\
	Skype:   Website: \\	


	\noindent\rule{16cm}{0.4pt}

\end {center}



			

{\bf RESEARCH INTERESTS} \\	
Intensity Frontier,  GUT and Flavor Physics. Data Analysis and Monte Carlo Simulation.  Machine learning.  \\

{\bf EDUCATION} 

{\bf Phd in Physics}, \href{http://physics.yale.edu/}{\textbf{Yale University}}   \hfill expected defense: Spring 2018 \\
Advisors: Prof Bonnie Fleming\footnote{Yale University}, Prof Flavio Cavanna\footnotemark[1] \\	
%Area of Study: Particle Physics, Neutrino Physics, Hadron Cross Sections, LArTPC   \\


\textbf{M.S. in  Nuclear and Particle Physics}, \href{http://www.scienze.unibo.it/}{\textbf{University of Bologna}}, \hfill  March 23$^{rd}$ 2012  \\
Thesis Title: Study of low p$_{T}$ D$^0$ meson production at CDF II in $p \bar p$ collisions at $\sqrt s$ = 900 GeV.  \\
Advisor: Professor Stefano Zucchelli\footnote{Department of Physics and Astronomy, University of Bologna, Bologna\label{ercolessi}}  \\
%Area of Study: Particle Physics, Flavor Physics, Charm Production in Collisions \\

%\href{http://www.scienze.unibo.it/}{\textbf{University of Bologna}},  Bologna, Italy\\ 
\textbf{B.S. in Physics}, \href{http://www.scienze.unibo.it/}{\textbf{University of Bologna}},   \hfill  December 11$^{th}$ 2009  \\
Thesis Title: Optimization of the reconstruction of neutrino interaction vertices in OPERA experiment. \\
Advisor:  Professor Maximiliano Sioli\footnotemark[2]  \\
%Area of study: Neutrino Physics, Data Analysis.  \\
\end{figure}



\begin{figure}[h]
{\bf AWARDS} 

\begin{itemize}
{\fontsize{10pt}{10pt} \selectfont
\item[-]Best poster at the 2017 International Neutrino Summer School, INSS \hfill 2017
\item[-]Winner of URA Visiting Scholar Program Award for the work \\``Study of nucleon decay topologies and their background in LArTPCs", URA \hfill2015
\item[-]Winner of Leigh Page Prize, Yale University \hfill2013
\item[-]Winner of the scholarship for the deepening of an international thesis, \\University of Bologna \hfill2012
\item[-]Winner of the scholarship for international thesis, University of Bologna  \hfill2011
\item[-]Placed 3$^{rd}$ in the contest �Inventare il futuro�, University of Bologna \hfill2011
\item[-]Winner of the scholarship ``Orfani Enasarco", Enasarco Foundation \hfill2010, 2008-2006, 2004, 2003}
\end{itemize}
\end{figure}

\newpage
\begin{center}
	{\large\bf Elena Gramellini} \\
	\noindent\rule{16cm}{0.2pt}
\end {center}


{\bf RESEARCH EXPERIENCE} 
\begin{itemize}
\item {\bf Hardware} 
\begin{itemize}
	\item [*] MicroBooNE CRT system
	\begin{itemize}	
		\item [-] Testing and Installation of 73 CRT modules
		\item [-] CRT Module Construction 
		\item [-] Trouble shooting of CRT Front End Board electronics
		\item [-] Design, management and installation of the cable connections for the entire system
		\item [-] Design of the Near Line CRT Metadata Storage in Fermilab file-system
		\item [-] Currently serving as CRT expert within the MicroBooNE Collaboration		
	\end{itemize}
	\item [*] Study of a Cherenkov Detector and a Muon Range Stack for the LArIAT Run I beam line.
	\item [*] Participated in the assembly and testing of the LArIAT Run III TPC.
\end{itemize}

\item {\bf Simulation} 
\begin{itemize}
	\item [*] Head of the LArIAT simulation production team 
	\item [*] Design and implementation of the ``Data Driven Monte Carlo" event generator for the LArIAT beam line  
\end{itemize}


\item {\bf Analysis} 
\end{itemize}

{\bf TEACHING EXPERIENCE} 

{\bf TALKS,  POSTERS  \& PRESENTATIONS}

I present regularly at the MicroBooNE and LArIAT Collaboration meetings and group meetings. 
\indent What follows is the list of presentations given outside my collaborations.

\begin{itemize}	
{\fontsize{10pt}{10pt} \selectfont
	\item[-]   INSS 2017 (Poster), $Fermilab$, $IL$  \hfill August 2017 \\
	``A study of the inclusive hadronic kaon-argon interaction cross section" 
	\item[-]  DPF 2017 (Talk), $Fermilab$, $IL$   \hfill August 2017 \\
	``A study of the inclusive hadronic kaon-argon interaction cross section"
	\item[-]  Joint SBN-DUNE Meeting (Talk), $Fermilab$, $IL$ \hfill May 2017 \\
	``MuCS measurements and CRT measurements" 
	\item[-]  ICHEP 2016 (Poster),  $Chicago$, $IL$  \hfill August 2016 \\
	``A MC study of kaon identification sensitivity in MicroBooNE"                       
	\item[-] ICHEP 2016 (Poster),  $Chicago$, $IL$ \hfill August 2016 \\
	``Study of the positive kaon total interaction cross section on Ar in LArIAT"
	\item[-] Yale WIDG Seminar (Seminar), $New$ $Haven$, $CT$	\hfill May 2016 \\
	``LArIAT - Liquid Argon In A Testbeam - Total $\pi-Ar$ cross section measurement"
	\item[-]   TAUP2015 (Talk), $Turin$, $Italy$  \hfill September 2015 \\
	``Studies of cosmogenic background to nucleon decay in MicroBooNE``                  
	\item[-]  New Perspectives 2015 (Talk), $Fermilab$, $IL$  \hfill June 2015 \\
	``LArIAT - Liquid Argon In A Testbeam`` 
	\item[-] CIPANP2015 (Talk),  $Veil$, $CO$	  \hfill May 2015 \\
	``LArIAT - Liquid Argon In A Testbeam``
	\item[]  }
\end{itemize}


{\bf PUBLICATIONS/MANUSCRIPTS IN PREPARATION} 

As a member of the MicroBooNE, LArIAT and CDF Collaborations, I am co-author of  $\sim$ 70 articles. 
\indent What follows is a list of selected publications.


\begin{itemize}
	{\fontsize{10pt}{10pt} \selectfont
	\item [-] ``A Novel Cosmic Ray Tagger System for Liquid Argon TPC Neutrino Detectors"\\	
	Martin Auger {\it et al.}\\
	Instruments, DOI: 10.3390/instruments1010002
	
	 \item[-]  ``LArIAT: Liquid Argon In A Testbeam''\\
            J.Paley {\it et al.}  [LArIAT Collaboration],\\
	   arXiv:1406.5560
	   \item[]}
\end{itemize}

{\bf SKILLS} 

\begin{itemize}	
	{\fontsize{10pt}{10pt}\selectfont	
	\item [-] Programming/scripting languages: C/C++, Python.
	\item [-] Simulation packages: GEANT4, Genie.
	\item [-] Data analysis: ROOT (C++), PyROOT, samweb.
	\item [-] Other Software: Art \& LArSoft,  \LaTeX, Mathematica, Office Package, Photoshop.
	\item [-] Experience with MySQL, xml.
	\item [-] Operating systems: Linux and Unix-based Operating Systems.	
}
\end{itemize}	


{\bf OUTREACH \& COMMUNITY} 
\begin{itemize}	
{\fontsize{10pt}{10pt} \selectfont
	\item[-] Elected member of the Climate and Diversity Committee\\ for the Yale physics department \hfill  2014-ongoing
	\item[-] MicroBooNE tour guide \hfill  2015-ongoing
	\item[-] Speaker at the TechSavvy initiative for middle school girls in STEM \hfill  March 2017
	\item[-] PechaKucha Speaker at the Batavia PechaKucha night Vol.6  \hfill  February 2017	
	\item[-] Virtual Reality tour guide for the Fermilab Family Open House \hfill   February 2017
	\item[-] Participant to the DUNE outreach initiative ``We are DUNE " \hfill   February 2017
	\item[-] Virtual Reality tour guide for the Fermilab Family Open House \hfill   February 2017
	\item[-] Fermilab Students and Postdocs Association elected fellow \hfill 2015-2016
\begin{itemize}	
\item[*] Head organizer of the 2016 New Perspectives conference
\item[*] Participant to the Fermilab Congressional Visit 2016
\end{itemize}		
	\item[-] Participant to the Fermilab outreach initiative ``Why I love Neutrinos " \hfill  December 2015
	\item[-] Facilitator in the Yale Physics Olympics for high school students \hfill  2014
	\item[-] Participant to the international art workshop \\``Sing, dance, paint to open your heart� promoted by the European Union \hfill  2005
	\item[-] Professional basketball player  \hfill 2005-2006
}
\end{itemize}	

{\bf REFERENCES} 

\begin{itemize}	
	{\fontsize{10pt}{10pt}\selectfont	
	\item[] {\bf Prof. Bonnie Fleming}\\ \href{mailto:bonnie.fleming@yale.edu}{bonnie.fleming@yale.edu}\\
		PhD advisor, MicroBooNE Co-Spokesperson. Fermilab \& Yale University.
	\item[]{\bf Prof. Flavio Cavanna} \\\href{mailto:flavio.cavanna@yale.edu}{flavio.cavanna@yale.edu}\\
		PhD advisor. Fermilab \& Yale University.
	\item[]{\bf Dr. Sam Zeller} \\ \href{mailto:gzeller@fnal.gov}{gzeller@fnal.gov}\\
                     MicroBooNE Co-Spokesperson. Fermilab.
	\item[]{\bf Dr. Jennifer Raaf}\\ \href{mailto:jlraaf@fnal.gov}{jlraaf@fnal.gov}\\
		LArIAT Co-Spokesperson. Fermilab.
	\item[] {\bf Prof. Jonathan Asaadi}\\ \href{jonathan.asaadi@uta.edu}{jonathan.asaadi@uta.edu}\\
		LArIAT Co-Spokesperson. UTA.	
	\item[]}
\end{itemize}

\end{document}
