\documentclass[10pt]{article}
\usepackage[applemac]{inputenc}
\usepackage{bbm,epigraph,cmbright}
\usepackage[english]{babel}
\usepackage{amssymb,amsmath,amsfonts}
\usepackage{calc}
\reversemarginpar
\usepackage[paper=a4paper,
            marginparwidth=30.5mm,    % Length of section titles
            marginparsep=1.mm,       % Space between titles and text
            margin=25mm,              % 25mm margins
            includemp]{geometry}
\def\changemargin#1#2{\list{}{\rightmargin#2\leftmargin#1}\item[]}
\let\endchangemargin=\endlist 
\setlength{\parindent}{0in}
\usepackage{eurosym}
\usepackage{paralist}
\usepackage{fancyhdr,lastpage}
\pagestyle{fancy}
\fancyhf{}\renewcommand{\headrulewidth}{0pt}
\fancyfootoffset{\marginparsep+\marginparwidth}
\newlength{\footpageshift}
\setlength{\footpageshift}
          {0.5\textwidth+0.5\marginparsep+0.5\marginparwidth-2in}
\lfoot{\hspace{\footpageshift}%
       \parbox{4in}{\, \hfill %
                    \arabic{page} of \protect\pageref*{LastPage} % +LP
                    \hfill \,}}
\usepackage{color,hyperref}
\definecolor{darkblue}{rgb}{0.0,0.0,0.3}
\hypersetup{colorlinks,breaklinks,
            linkcolor=darkblue,urlcolor=darkblue,
            anchorcolor=darkblue,citecolor=darkblue}
\newcommand{\makeheading}[1]%
        {\hspace*{-\marginparsep minus \marginparwidth}%
         \begin{minipage}[t]{\textwidth+\marginparwidth+\marginparsep}%
                {\large \bfseries #1}\\[-0.15\baselineskip]%
                 \rule{\columnwidth}{1pt}%
         \end{minipage}}
\newcommand{\Nsection}[2]%
        {\pagebreak[2]\vspace{1.3\baselineskip}%
         \phantomsection\addcontentsline{toc}{section}{#1}%
         \hspace{0in}%
         \marginpar{
         \raggedright \scshape #1}#2}
\newenvironment{outerlist}[1][\enskip\textbullet]%
        {\begin{itemize}[#1]}{\end{itemize}%
         \vspace{-.8\baselineskip}}
\newenvironment{lonelist}[1][\enskip\textbullet]%
        {\vspace{-\baselineskip}\begin{list}{#1}{%
        \setlength{\partopsep}{0pt}%
        \setlength{\topsep}{0pt}}}
        {\end{list}\vspace{-.6\baselineskip}}
\newenvironment{innerlist}[1][\enskip\textbullet]%
        {\begin{compactitem}[#1]}{\end{compactitem}}
\newcommand{\blankline}{\quad\pagebreak[2]}
%\usepackage[none]{hyphenat}


\hyphenation{phe-no-me-no-lo-gi-cal}







\begin{document}

\makeheading{Elena Gramellini\hfill CV}

\Nsection{Contact Information}
\newlength{\rcollength}\setlength{\rcollength}{1.85in}%
\begin{tabular}[t]{@{}p{\textwidth-\rcollength}p{\rcollength}}
\emph{Currently at:} \href{http://physics.yale.edu/}%
     {Department of Physics}  
\href{http://physics.yale.edu/}{Yale University} & \\
\emph{Web:}  \href{https://www.elenagramellini.com}{elenagramellini.com} &\\ 
\textit{E-mail:}  \href{mailto:elena.gramellini@yale.edu}{elena.gramellini@yale.edu} & %
\end{tabular}

\Nsection{Research Interests}
Intensity Frontier \& Neutrino Physics, GUT and Flavor Physics.  \\Liquid Argon Neutrino Detectors. Machine learning.  \\

\Nsection{Education}
\href{http://physics.yale.edu/}{\textbf{Yale University}}, 
New Haven, CT, USA 
\begin{outerlist}

\item[] \textbf{Phd candidate}, \hfill  August 2013 - current
	\begin{innerlist}
	\item[] Advisors: Prof. Bonnie Fleming\footnote{Yale University \& Fermilab}, Prof. Flavio Cavanna\footnotemark[1]
	\item[] Area of Study: Neutrino Physics, Hadron Cross Sections, LArTPC
	\end{innerlist}
\end{outerlist}
\vspace{0.5cm}


\href{http://www.scienze.unibo.it/}{\textbf{University of Bologna}}, 
Bologna, Italy
\begin{outerlist}
\item[] \textbf{M.S. in  Nuclear and Particle Physics},  \hfill March 2012
	\begin{innerlist}
	\item[] Thesis Title: Study of low p$_{T}$ D$^0$ meson production cross section\\ at CDF II in $p \bar p$ collisions at $\sqrt s$ = 900 GeV.
	\item[] Advisor: Prof. Stefano Zucchelli\footnote{Department of Physics and Astronomy, University of Bologna, Bologna\label{ercolessi}}
	\item[] Area of Study: Flavor Physics, Charm Production at Colliders
	\end{innerlist}
	
\item[] \textbf{B.S. in Physics},  \hfill December 2009
        \begin{innerlist}
        \item[] Thesis Title: Optimization of the vertex reconstruction \\in OPERA neutrino interaction events.
        \item[] Advisor: Prof. Maximiliano Sioli\footnotemark[2]
        \item[] Area of study: Neutrino Physics, Data Analysis
        \end{innerlist}
\end{outerlist}
\vspace{0.5cm}

\textbf{Professional training}
\begin{innerlist}
\item[] International Neutrino Summer School \hfill August 2017
\item[]  Alan Alda Center scientific communication workshop  \hfill May 2017
\item[]  APS Short Course on Nuclear Weapon and Related Security Issues  \hfill April 2017
\item[] Phystat-Nu Fermilab, statistics for neutrino physics workshop \hfill September 2016 
\item[] Fermilab Summer School \hfill Summer 2010
\end{innerlist}



\Nsection{Awards \& scholarships}
\begin{outerlist}
\item[] 2017-2018 Dean�s Emerging Scholars Research Award  \hfill 2017\\
\emph{Yale Office of the Provost and Graduate School of Arts and Sciences}
\item[]Best poster at the 2017 International Neutrino Summer School, INSS \hfill 2017\\
\emph{A study of charged kaon-nucleon total interaction cross section }
\item[]URA Visiting Scholar Program Award   \hfill2015\\
\emph{Award for the peer-reviewed proposal}\\\emph{``Study of nucleon decay topologies and their background in LArTPCs"}
\item[]Leigh Page Prize, Yale University  \hfill2013\\
\emph{Academic Based Award for Incoming Graduate Students}
\item[]Scholarship for international thesis   \& \hfill2011-2012\\
Scholarship for the deepening of an international thesis, \\
\emph{University of Bologna Awards for the peer-reviewed proposal}\\\emph{``Measurement of D$^0$ production cross section at the CDF experiment"}

\item[]Placed 3$^{rd}$ in the contest �Inventare il futuro,\hfill2011\\ 
\emph{University of Bologna Awards for a peer-reviewed proposal}\\ \emph{on technology applications to welfare}
\item[]Scholarship ``Orfani Enasarco", \hfill2010, 2008-2006, 2004, 2003\\
Enasarco Foundation 
\end{outerlist}
	
	

	
\Nsection{Committee \& Academic Service}
\begin{outerlist}
	\item[-] Fermilab Students and Postdocs Association elected fellow \hfill 2015-2016
\begin{innerlist}	
\item[+] Head organizer of the 2016 New Perspectives conference
\item[+] Regular participation to the Fermilab User Executive Committee
\item[+] Participant in the Fermilab Visit to the US Congress 2016
\end{innerlist}		

\item[-] Elected member of the Climate and Diversity Committee \hfill  2014-ongoing\\ for the Yale physics department

\end {outerlist}



%\newpage
\Nsection{Teaching \& Mentoring}
\begin{outerlist}
	\item [-] Mentored Students: William De Rocco (undergrad), Daniel Smith (undergrad), Marina Guzzo (master student), Ohana Rodrigues (master student), Supraja Balasubramanian (PhD candidate).
	\item [-] LArSoft, Grid and Data Handling tutorials \hfill2014-current\\for the LArIAT Collaboration \& LArSoft users
	\item [-] Teaching Fellow, Lab Instructor for P165, Yale University \hfill2014-2015
	\item [-] Teaching Fellow, Discussion Leader for P180-P181, Yale University \hfill2013-2014
	\item [-] Teaching Fellow for the �Fisica t-a� (General Physics) class  \hfill2012-2013 \\ in mechanical engineering , University of Bologna 
%	\item [-] Math and Physics tutoring for Middle school and high school students \hfill2011-2013
\end {outerlist}



%\end{outerlist}
\Nsection{Talks, Posters \& Presentations}
%I present regularly at the MicroBooNE and LArIAT Collaboration meetings and group meetings. What follows is the list of presentations given outside my collaborations.\\
\textbf{Invited talks:}
\begin{outerlist}
	\item[-] Physics Department Seminar, \emph{University of Bologna \& INFN}, $Italy$	\hfill Upcoming \\
	``Liquid Argon detectors for Neutrino Physics @FNAL"
	\item[-] High Energy Physics Group Seminar, \emph{Imperial College London}, $UK$	\hfill Nov 2017 \\
	``Liquid Argon under investigation: first results from the LArIAT experiment"
	\item[-] 2$^{nd}$ UK LArSoft Workshop, \emph{Manchester University}, $UK$	\hfill Oct 2017 \\
	 ``LArSoft Architecture, MC and Grid Submission"
	\item[-]  Joint SBN-DUNE Meeting,  \emph{Fermilab}, $IL$ \hfill May 2017 \\
	``MuCS measurements and CRT measurements" 
	\item[-] Niel Bohr Lunch Seminar, \emph{Manchester University}, $UK$	\hfill Jan 2017 \\
	``Liquid Argon under investigation: first results from the LArIAT experiment"
	\item[-]  WIDG Seminar, \emph{Yale University}, New Haven, $CT$	\hfill May 2016 \\
	``LArIAT: Total $\pi-Ar$ cross section measurement"
\end{outerlist}
\vspace{0.5cm}

\textbf{Other Contributions at International Conferences:}
\begin{outerlist}
	\item[-]  DPF 2017 (Talk), $Fermilab$, $IL$   \hfill August 2017 \\
	``A study of the inclusive hadronic kaon-argon \\interaction cross section"
	\item[-]   INSS 2017 (Poster), $Fermilab$, $IL$  \hfill August 2017 \\
	``A study of the inclusive hadronic kaon-argon \\interaction cross section" 
	\item[-]  ICHEP 2016 (Poster),  $Chicago$, $IL$  \hfill August 2016 \\
	``A MC study of kaon identification sensitivity in MicroBooNE"                       
	\item[-] ICHEP 2016 (Poster),  $Chicago$, $IL$ \hfill August 2016 \\
	``Study of the positive kaon total interaction cross section\\ on Ar in LArIAT"
	\item[-]   TAUP2015 (Talk), $Turin$, $Italy$  \hfill September 2015 \\
	``Studies of cosmogenic background to nucleon decay\\ in MicroBooNE``                  
	\item[-]  New Perspectives 2015 (Talk), $Fermilab$, $IL$  \hfill June 2015 \\
	``LArIAT - Liquid Argon In A Testbeam`` 
	\item[-] CIPANP2015 (Talk),  $Veil$, $CO$	  \hfill May 2015 \\
	``LArIAT - Liquid Argon In A Testbeam``
\end{outerlist}





\Nsection{Outreach \& Interpersonal Skills}
I regularly participate in outreach activities: to see some of my contributions,  please visit \href{https://www.elenagramellini.com/outreach}{www.elenagramellini.com/outreach}.
\begin{outerlist}
	\item[-] MicroBooNE tour guide \hfill  2015-ongoing
	\item[-] Ask-A-Scientist participant for the 2017 Fermilab Open House \hfill September 2017
	\item[-] Speaker at the TechSavvy initiative for middle school girls in STEM \hfill  March 2017
	\item[-] PechaKucha Speaker at the Batavia PechaKucha night Vol.6  \hfill  February 2017	
	\item[-] Participant to the DUNE outreach initiative ``We are DUNE " \hfill   February 2017
	\item[-] Virtual Reality tour guide for the Fermilab Family Open House \hfill   February 2017
	\item[-] Fermilab tour guide for the \\National Society of Black Physicists conference \hfill October 2016
	\item[-] Panelist for Discussion with Students from Rwanda (ICHEP2016) \hfill August 2016
	\item[-] Presenter for   Wicked Science ``STEM and Girls" initiative \hfill June 2016
	\item[-] Participant to the Fermilab outreach initiative\\ ``Why I love Neutrinos " \hfill  December 2015
	\item[-] Facilitator in the Yale Physics Olympics for high school students \hfill April 2014
	\item[-] Tour guide for the exposition "The energy of the void" (INFN) \hfill   January 2013


	\item[-] Professional basketball player  \hfill 2005-2006
\end {outerlist}


\Nsection{Programming \& Computing Skills}
\begin{outerlist}
	\item [-] Programming languages: C/C++, Python.
	\item [-] Simulation packages: GEANT4, Genie.
	\item [-] Data analysis: ROOT (C++), PyROOT, Art \& LArSoft, Samweb.
	\item [-] Other Software:   \LaTeX, Mathematica, Photoshop.
	\item [-] Data base experience with MySQL, xml.
	\item [-] Operating systems: Linux and Unix-based Operating Systems.	
\end {outerlist}
\blankline


\blankline


\end{document}

%%%%%%%%%%%%%%%%%%%%%%%%%% End CV Document %%%%%%%%%%%%%%%%%%%%%%%%%%%%%
